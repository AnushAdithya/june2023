\documentclass[14pt]{article}
\usepackage{amsmath}

\date{June 2023}

\begin{document}
\author{\Large{Bharath Sajeev}}
\title{\LARGE{\textbf{\underline{Nernst Equation}}}}
\maketitle

\section{CH22B012}
\textbf{Name :Bharath Sajeev}

\noindent \textbf{Github User-id :Bharath-Sajeev}

\vspace{0.5cm}

\noindent In  electrochemistry, the Nernst equation is a chemical thermodynamical relationship that permits the calculation of the reduction potential of a reaction (half-cell or full cell reaction) from the standard electrode potential, absolute temperature, the number of electrons involved in the redox reaction, and concentrations of the chemical species undergoing reduction and oxidation respectively. It was named after Walter Nernst, a German physical chemist who formulated the equation.

\vspace{0.5cm}


\begin{equation} \label{NernstEquation} \footnote{https://en.wikipedia.org/wiki/Nernst-equation} 
E_{cell} = E^0_{cell} - \frac{RT}{zF} \ln{Q_r}
\end{equation}
where:


$ E_{cell} $ is the cell potential at the temperature of interest 


$ E^0_{cell} $ is the standard cell potential 

$R$ is the universal gas constant:$R=8.314 J K^{-1} mol^{-1}$

$T$ is the temperature in kelvins

$z$ is the number of electrons transferred in the cell reaction

$F$ is the Faraday constant:$F=96,485 C mol^{-1}$

$Q_r$ is the reaction quotient of the cell reaction

\vspace{1cm}

\noindent {\Large{\textbf{Why I chose this equation}}}

\vspace{0.5cm}

\noindent The reason I chose this equation is because I love chemistry and I have very fond memories about solving questions using this equation during my JEE days. It was fun to write out and balance the chemical reactions and find the cell potential.
\end{document}
