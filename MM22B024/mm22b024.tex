\documentclass{article}
\usepackage{graphicx} % Required for inserting images
\usepackage{amsmath}



\title{\textbf{Taylor's Series}}
\author{Chirag V Gupta}
\date{MM22B024}

\begin{document}
\maketitle
\section{MM22B024}
\subsection{Details}
\textbf{Name}: \textit{Chirag Vikram Gupta} \\
\textbf{Git-Hub User ID}: \textit{CVG-2904}
\subsection{Formulae}
The Taylor series expansion of a function $f(x)$ around $x=a$ is given by:

\[
f(x) = f(a) + \frac{{f'(a)}}{{1!}}(x-a) + \frac{{f''(a)}}{{2!}}(x-a)^2 + \frac{{f'''(a)}}{{3!}}(x-a)^3 + \ldots
\] \footnote{MA1102 Classnotes
Series and Matrices, Chapter 2 Section 2.3}

Alternatively, the compact form using sigma notation is:

\[
f(x) = \sum_{n=0}^{\infty} \frac{{f^{(n)}(a)}}{{n!}}(x-a)^n
\]\footnote{MA1102 Classnotes
Series and Matrices, Chapter 2 Section 2.3}

\subsection{Taylor Polynomials}
The linearization of a differentiable function ƒ at a point a is the polynomial of degree one
given by

\[P1(x) = f(a) + f'(a)(x-a).\]
\footnote{Thomas Calculus}
If f has derivatives of higher order at a, then it has higher-order polynomial approximations as well, one for each available derivative. These polynomials are called the Taylor polynomials of f. \\
We speak of a Taylor polynomial of order n rather than degree n because ƒ(n)
(a) may
be zero. The first two Taylor polynomials of ƒ(x) = cos x at x = 0, for example, are
P0(x) = 1 and P1(x) = 1. The first-order Taylor polynomial has degree zero, not one.
Just as the linearization of ƒ at x = a provides the best linear approximation of ƒ in
the neighborhood of a, the higher-order Taylor polynomials provide the “best” polynomial
approximations of their respective degrees.






\end{document}