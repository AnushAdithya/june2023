\documentclass{article}
\usepackage{graphicx} % Required for inserting images
\usepackage{amsmath}
\usepackage{hyperref}

\title{Assignment 4}
\author{Akanksh C J}
\date{}

\begin{document}

\maketitle

\begin{center} \section*{CH22B010} \subsection*{Fundamental Theorem of Calculus} \end{center}
\begin{itemize} \item \textbf{Formula:} 
{\Large $$ \int_a^bf'(x)dx = f(b) - f(a) $$}
\end{itemize}

\par The fundamental theorem of calculus, as the name suggests is the most basic and fundamental concept in the domain of calculus. It relates two of the most important operations in mathematics, integration and differentiation, as the inverse operations of each other.
\par The theorem says that if there is a function f(x), with its derivative as f'(x), then the integration of the derivative over the interval a to b, considering that the derivative is continuous and smooth over the interval, then the result is the same as taking the difference between the values of the function f(x) at these limiting values.
\par This theorem has great physical significance, relating two completely independent operations and is at the heart of the mathematical domain of calculus.

\begin{center} \subsection*{GitHub reference} \end{center}
\begin{itemize} \item Name: Akanksh C J
\item GitHub User ID: AkankshCJ
\end{itemize}
\footnote{The above equation was referred from \href{https://en.wikipedia.org/wiki/Fundamental_theorem_of_calculus}{Wikipedia}}
\end{document}
