\documentclass[a4paper, 12pt]{article}
\usepackage{amsmath}
\begin{document}

\section*{ID2090}

Assignment 4

\textbf{Name }: Prinal Kumat 

\textbf{RollNo. } : MM22B048

\textbf{Github user Id } :https://github.com/prinalkumat

\subsection*{Activation functions}
An activation function decides whether a neuron should be activated or not.
The purpose of the function is to introduce non-linearity into the output of the neuron.

\subsection{Sigmoid}
Related to 
\emph{Logistic Regression}  
For single label or multi-label binary classification	
\begin{equation}
\sigma(z) = \frac{1} {1 + e^{-z}}
\end{equation}

\subsection{Tanh}
\begin{equation}
tanh(x) = \frac{e^x - e^{-x}}{e^{x} + e^{-x}} = \frac{1 - e^{-2x}}{1 + e^{-2x}}
\end{equation} 

In neural networks the process of 
\emph{back propogation}   
is possible because of 
activation functions, that is gradients are supplied along 
with the error to update the 
weights and biases.

\footnote{https://www.geeksforgeeks.org/activation-functions-neural-networks/}
 
\end{document}
