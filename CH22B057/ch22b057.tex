\documentclass{article}
\usepackage{graphicx}
\usepackage{amsmath}
\usepackage{url}

\begin{document}

\title{Kirchhoff's Laws}
\author{CH22B057}

\maketitle

\section{Introduction}
Kirchhoff's Laws are fundamental principles in electrical circuit analysis. They were formulated by German physicist Gustav Kirchhoff in the mid-19th century. These laws are widely used to analyze and solve complex electrical circuits.

\section{Kirchhoff's Current Law}
This law, also called Kirchhoff's first law, or Kirchhoff's junction rule, states that, for any node (junction) in an electrical circuit, the sum of currents flowing into that node is equal to the sum of currents flowing out of that node; or equivalently: 
      The algebraic sum of currents in a network of conductors meeting at a point is zero.
Recalling that current is a signed (positive or negative) quantity reflecting direction towards or away from a node, this principle can be succinctly stated as: 

\begin{equation}
\sum_{i} I_i = 0
\end{equation}
where $\sum_{i}$ represents the sum of currents entering the junction, and $I_i$ represents each individual current.~ref{eqn:enwiki:1160647815}

KCL is based on the principle of conservation of charge, which states that charge cannot be created or destroyed within an electrical circuit. Therefore, any charge that enters a junction must also leave it.

\section{Kirchhoff's Voltage Law}
Kirchhoff's Voltage Law, also known as the loop rule, states that the sum of all voltages around any closed loop in a circuit is equal to zero. This law is based on the principle of conservation of energy.

In an electrical circuit, voltage represents the electrical potential difference between two points. KVL can be mathematically expressed as:
\begin{equation}
\sum_{i} V_i = 0
\end{equation}
where $\sum_{i}$ represents the sum of voltages around a closed loop, and $V_i$ represents each individual voltage.

KVL is derived from the principle that the total work done by an electric field on a charge in a closed loop must be zero. This law is crucial for analyzing circuit elements such as resistors, capacitors, and inductors.

\section{Application of Kirchhoff's Laws}
Kirchhoff's Laws are powerful tools for analyzing and solving complex electrical circuits. They provide a systematic approach to circuit analysis and allow engineers to determine the behavior of circuits under various conditions.

Some common applications of Kirchhoff's Laws include:
\begin{itemize}
\item Solving for unknown currents or voltages in a circuit.
\item Analyzing circuits with multiple loops and junctions.
\item Determining the power dissipated in different circuit elements.
\item Designing and optimizing electrical circuits.
\end{itemize}

\section{Conclusion}
Kirchhoff's Laws are fundamental principles in electrical circuit analysis. They provide a mathematical framework for understanding and solving complex electrical circuits. By applying Kirchhoff's Current Law and Kirchhoff's Voltage Law, engineers and scientists can analyze and design circuits for a wide range of applications.

\section{References}
This information is referred from Wikipedia : Kirchoffs Law

\bibliography{newfile}
\bibliographystyle{plain}

\end{document}


