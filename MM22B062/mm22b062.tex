\documentclass{article}
\usepackage{cite}
\usepackage{graphicx}

\title{Van der Waals equation}
\author{ROLL NO. : MM22B062 \\ NAME : TANISH MESHRAM \\ Github ID : Tanishq02005}
\date{June 2023}

\begin{document}

\maketitle

\section*{MM22B062}

For this assignment, I have choosen the \texttt{'Van der Waals equation'}which is expressed as :

\begin{equation}
\left(P + \frac{an^2}{V^2}\right) \left(V - bn \right) = nRT
\end{equation}

where , 
\begin{center}
    $P$ = Pressure of the gas
\end{center}
\begin{center}
    $T$ = Temperature of the gas
\end{center} 
\begin{center}
    $V$ = Volume of the gas
\end{center}
\begin{center}
    $R$ = Gas constant
\end{center}
\begin{center}
    $n$ = No. of moles
\end{center}
\begin{center}
    '$a$' and '$b$' are the Van der Waals constant specific to each gas
\end{center} 

According to \footnote{The book 'Thermodynamics: An Engineering Approach' by 'Yunus A. Cengal and Michael A. Boles' in 'Chapter 3.Properties of Pure Substances' Page no. 142}, the  \texttt{Van der waals equation} has a historical value in that, it was one of the first attempts to model the behaviour of real gas.

To get this equation, we have to make some corrections in our Ideal gas equation which is : 

\begin{equation}
PV = nRT
\end{equation}

To account for the volume occupied by the real gas molecules, the \texttt{Van der waals equation} replaces V/n in the ideal gas equation with ($V_{m}$ - $b$), where $V_{m}$ is the molar volume of the gas and $b$ is the volume occupied by the molecules of one mole.The second modification we have to made is for the intermolecular forces. The \texttt{Van der waals equation} includes intermolecular forces by adding to the observed pressure in the equation of state a term of the form $a/(V_m)^2$, where '$a$' is a constant whose value vary from gas to gas.By this changes in the ideal gas equation, we get
\begin{equation}
    \left(P + \frac{a}{V^2}\right) \left(V - b \right) = RT
\end{equation}
for 1 mole and equation (1) as the complete \texttt{Van der waals equation} for 'n' no. of moles.

So this is how we get the \texttt{Van der waals equation}.

The determination of two constants $a$ and $b$ is based upon the observation that the critical isotherm on a $P-V$ diagram has a horizontal inflection point at the critical point.To determine the two constants,we apply that the first and the second derivative of $P$ with respect to $V$ at the critical point must be zero. 

\end{document}
