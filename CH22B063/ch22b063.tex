\documentclass{article}

\author{Devesh Negi CH22B063}
\title{	Gibbs-Helmholtz equation}

\begin{document}

\maketitle

\section{CH22B063}

Name: Devesh Negi \\
GitHub User-ID: DEVESH-N2 

\subsection{Description}

The Gibbs–Helmholtz equation is a thermodynamic equation used for calculating changes in the Gibbs free energy of a system as a function of temperature. It was originally presented in an 1882 paper entitled "Die Thermodynamik chemischer Vorgänge" by Hermann von Helmholtz. It describes how the Gibbs free energy, which was presented originally by Josiah Willard Gibbs, varies with temperature.


The Gibbs-Helmholtz equation is an important thermodynamic relationship that relates the change in the Gibbs free energy ($\Delta G$) of a system with respect to temperature ($T$). It provides insights into the temperature dependence of a chemical reaction or a physical process. The equation is expressed as follows:

\[
\frac{\partial(\Delta G/T)}{\partial T} = -\frac{\Delta H}{T^2}
\]

The equation indicates that the partial derivative of the change in Gibbs free energy divided by temperature with respect to temperature is equal to the negative of the change in enthalpy divided by the square of the temperature.


This equation provides insights into the temperature dependence of the Gibbs free energy and can be used to analyze the thermodynamic stability of a system. Specifically, it helps determine how the Gibbs free energy changes as temperature varies. A negative value for the partial derivative indicates that the Gibbs free energy decreases as the temperature increases, suggesting a more favorable and stable system.


\subsection{Applications of Gibbs-Helmholtz equation}
The applications of Gibbs Helmholtz equation are as follows:

\begin{itemize}

  \item It can be used to determine the overall enthalpy of a reaction and its variation with temperature from the given value of Gibbs free energy at constant pressure.
  \item It can be used to determine the value of Gibbs free energy for a reaction at a temperature other than 298 K.
  \item It can be used to determine the effect of change in temperature on equilibrium constant.
  \item It can be used to determine the spontaneity of a reaction.

\end{itemize}

\subsection{Conclusion}
The Gibbs-Helmholtz equation finds applications in various fields of study, including chemical reactions, phase transitions, and the analysis of thermodynamic equilibrium. It provides a quantitative tool to understand the thermodynamic behavior of systems as temperature changes.


By utilizing the Gibbs-Helmholtz equation, scientists and engineers can gain valuable insights into the thermodynamic stability and spontaneity of processes, aiding in the design and optimization of chemical reactions and other thermodynamic systems.

\begin{thebibliography}{1}
\bibitem{atkins} Atkins, P. W., de Paula, J. (Year). \textit{Physical Chemistry}. Publisher.
\end{thebibliography}


\end{document}

