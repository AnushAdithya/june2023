\documentclass{article}
\usepackage{hyperref}

\begin{document}

\section*{AE22B070}

\textbf{Name:} Chirag V Gaonkar \\
\textbf{GitHub User ID:} CVGaonkar

\subsection*{Remainder Theorem}

\textbf{Statement:}
The Remainder Theorem states that the remainder of any polynomial \(f(x)\) when divided by a linear polynomial \((x-a)\) is equal to the value of the function at the root of the divisor polynomial,
 i.e., the remainder is equal to \(f(a)\).

\textbf{Proof:}
The polynomial remainder theorem follows from the theorem of Euclidean division, which, given two polynomials \(f(x)\) (the dividend) and \(g(x)\) (the divisor), asserts the existence (and uniqueness)
 of a quotient \(Q(x)\) and a remainder \(R(x)\) such that \(f(x) = g(x) \cdot Q(x) + R(x)\). In the case of division by a linear polynomial, the remainder is a constant. So we can rewrite the above
 equation as \(f(x) = (x-a) \cdot Q(x) + R\). Now we can clearly see that when \(x=a\), this equation reduces to \(R = f(a)\).

\footnote{\url{https://en.wikipedia.org/wiki/Polynomial_remainder_theorem}}

\end{document}
