\documentclass{article}

\usepackage{amsmath}
\usepackage{amssymb}
\usepackage{amsthm}

\usepackage{graphicx}
\usepackage{hyperref}

\title{Stokes' Theorem}
\author{Charukhesh - AE22B028}
\date{}

\newtheorem{theorem}{Theorem}

\begin{document}

\maketitle

\begin{abstract}
This document presents an overview of Stokes' theorem, a fundamental result in vector calculus. Stokes' theorem relates the surface integral of a vector field over a closed surface to the line integral of the same vector field around the boundary curve of that surface. The theorem has wide applications in physics, engineering, and mathematics.
\end{abstract}

\section{AE22B028}

NAME : CHARUKHESH B R \\
GITHUB USER-ID : Charukhesh

\section{Introduction}

The concept of Stokes' theorem emerged from the work of George Gabriel Stokes, an Irish mathematician and physicist in the 19th century. Stokes' theorem is a generalization of Green's theorem, which relates the line integral of a vector field to a double integral over a region in the plane.

\section{Curl of a Vector Field}

Suppose that $\mathbf{F}$ is the velocity field of a fluid flowing in space. Particles near the point $(x, y, z)$ in the fluid tend to rotate around an axis through $(x, y, z)$ that is parallel to a certain vector we are about to define. This vector points in the direction for which the rotation is counterclockwise when viewed looking down onto the plane of the circulation from the tip of the arrow representing the vector. This is the direction your right-hand thumb points when your fingers curl around the axis of rotation in the way consistent with the rotating motion of the particles in the fluid. The length of the vector measures the rate of rotation. The vector is called the curl vector, and for the vector field $\mathbf{F} = M\mathbf{i} + N\mathbf{j} + P\mathbf{k}$, it is defined to be:

\[
\text{curl } \mathbf{F} = \left(\frac{{\partial P}}{{\partial y}} - \frac{{\partial N}}{{\partial z}}\right)\mathbf{i} + \left(\frac{{\partial M}}{{\partial z}} - \frac{{\partial P}}{{\partial x}}\right)\mathbf{j} + \left(\frac{{\partial N}}{{\partial x}} - \frac{{\partial M}}{{\partial y}}\right)\mathbf{k}.
\]

This information is a consequence of Stokes' Theorem, the generalization to space of the circulation-curl form of Green's Theorem and is the subject of this section.

Notice that $(\text{curl } \mathbf{F}) \cdot \mathbf{k} = \left(\frac{{\partial N}}{{\partial x}} - \frac{{\partial M}}{{\partial y}}\right)$, which is consistent with our definition in Section 16.4 when $\mathbf{F} = M(x, y)\mathbf{i} + N(x, y)\mathbf{j}$. The formula for curl $\mathbf{F}$ is often expressed using the symbol $\nabla$, pronounced "del":

\[
\text{curl } \mathbf{F} = \nabla \times \mathbf{F} = \begin{vmatrix} \mathbf{i} & \mathbf{j} & \mathbf{k} \\ \frac{{\partial}}{{\partial x}} & \frac{{\partial}}{{\partial y}} & \frac{{\partial}}{{\partial z}} \\ M & N & P \end{vmatrix}.
\]

We often use this cross product notation to write the curl symbolically as $\nabla \times \mathbf{F}$.

\section{Statement of Stokes' Theorem}

Stokes' theorem relates the flux of a vector field $\mathbf{F} = M\mathbf{i} + N\mathbf{j} + P\mathbf{k}$ across a surface $S$ to the circulation of $\mathbf{F}$ around the boundary curve $C$ in the counterclockwise direction with respect to the surface's unit normal vector $\mathbf{n}$. Mathematically, Stokes' theorem can be stated as follows:

\begin{theorem}[Stokes' Theorem]
Let $S$ be a piecewise-smooth, oriented surface with a piecewise-smooth boundary curve $C$. Let $\mathbf{F} = M\mathbf{i} + N\mathbf{j} + P\mathbf{k}$ be a vector field whose components have continuous first partial derivatives on an open region containing $S$. Then, the circulation of $\mathbf{F}$ around $C$ is equal to the surface integral of the curl vector field $\nabla \times \mathbf{F}$ over $S$:
\begin{equation}
\oint_C \mathbf{F} \cdot d\mathbf{r} = \iint_S (\nabla \times \mathbf{F}) \cdot \mathbf{n} \, d\sigma,~\cite{thomas2018calculus}
\footnote{
@book{ thomas2018calculus,
author = {George B Thomas, Joel Hass, Christopher Heil, Maurice D Weir},
title = {Calculus},
booktitle = {Thomas' Calculus},
year = {2018},
edition = {14},
publisher = {Pearson},
chapter = {16},
section = {16.7},
pages = {1019 - 1031}
}
}
\end{equation}
where $\nabla \times \mathbf{F}$ is the curl of $\mathbf{F}$, $\mathbf{n}$ is the unit normal vector to the surface $S$, $d\mathbf{r}$ is the differential vector along $C$, and $ds$ is the differential surface area element of $S$.
\end{theorem}

\section{Conclusion}

Stokes' Theorem provides a relationship between surface integrals and line integrals, enabling the conversion of one type of integral to another. It plays a crucial role in vector calculus and finds applications in various areas of mathematics and physics.

\bibliography{refs}
\bibliographystyle{alpha}

\end{document}
