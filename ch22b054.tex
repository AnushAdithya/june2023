\documentclass{article}
\usepackage{amsmath} % For mathematical equations and symbols
\begin{document}
\author{[CH22B054 \\ Anush Adithya]}
\title{Gibb's Phase Rule}
\maketitle
\date{June 19,2022}
The Gibbs phase rule is given by the equation: 
\begin{equation}  
    DF = 2 + c - n
\end{equation}

where:
$DF$ is the degree of freedom of the system. It represents the number of independent variables that can be varied without changing the number of phases in the system at equilibrium.
$c$ is the number of chemical species present in the system.
$n$ is the number of phases in the system at equilibrium.

The equation indicates that the degree of freedom ($DF$) is equal to the sum of two (2) plus the difference between the number of chemical species ($c$) and the number of phases ($n$). This formula allows us to determine the maximum number of independent variables that can be varied while keeping the system at equilibrium.

For example, consider a system with three chemical species ($c=3$) and two phases ($n=2$) at equilibrium. Substituting these values into the equation, we get:

\begin{align*}
    DF &= 2 + c - n \\
    &= 2 + 3 - 2 \\
    &= 3
\end{align*}

In this case, the system has three degrees of freedom ($DF=3$), which means that three independent variables can be varied without altering the equilibrium between the two phases.

The Gibbs phase rule is a fundamental concept in thermodynamics and is widely used to analyze and understand multi-component, multi-phase systems. It provides valuable insights into the behavior of such systems and helps in predicting their equilibrium conditions.

\end{document}
