\documentclass{article}

\begin{document}

\section{Coulomb's Law}

Coulomb's Law, named after the French physicist Charles-Augustin de Coulomb, is a fundamental principle in the study of electricity and magnetism. It describes the electrostatic force between two charged particles and helps us understand how electric charges interact.

The force between two point charges $q_1$ and $q_2$ is given by Coulomb's Law:

\[
F = \frac{{k \cdot |q_1 \cdot q_2|}}{{r^2}}
\]

where:
\begin{itemize}
  \item $F$ is the electrostatic force between the charges,
  \item $k$ is the electrostatic constant (also known as Coulomb's constant) and has a value of approximately $8.99 \times 10^9 \, \text{N m}^2/\text{C}^2$,
  \item $q_1$ and $q_2$ are the magnitudes of the charges, and
  \item $r$ is the distance between the charges.
\end{itemize}

The force is directly proportional to the product of the magnitudes of the charges and inversely proportional to the square of the distance between them. This means that the force decreases as the distance between the charges increases and increases as the magnitudes of the charges or the electrostatic constant increase.

Coulomb's Law enables us to calculate the force experienced by charged particles and understand the behavior of electric fields and electrically charged objects. It has wide-ranging applications in various fields, including physics, electrical engineering, and electronics.

It is important to note that Coulomb's Law holds for point charges, which are theoretical concepts used to simplify calculations. In real-world scenarios, charges may have spatial extent and other factors, such as the presence of other charges, may come into play.

Reference: Griffiths, D. J. (2017). Introduction to Electrodynamics. Cambridge University Press.

Name: Kushagra Kapoor \\
User ID: Kushagra-Kapoor-ae22b037

\end{document}
