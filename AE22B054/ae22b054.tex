\documentclass{article}

\title{Assignment 4}
\author{Rutika Gavit}
\date{}
\begin{document}
\maketitle
\section*{AE22B054}
\subsection*{The Doppler Effect}
The Doppler effect is a phenomenon observed in waves when there is relative motion between the source of the wave and the observer. It is named after the Austrian physicist Christian Doppler, who first described the effect in 1842. The Doppler effect can be observed in various waves, including sound, light, and ocean waves.

The Doppler effect can be understood by considering the compression or stretching of waves due to relative motion. When a wave source and an observer are moving toward each other, the effective wavelength of the waves appears shorter to the observer, resulting in a higher frequency. Conversely, when the source and the observer move away, the effective wavelength appears longer to the observer, resulting in a lower frequency.

Mathematically, the observed frequency $f_{obs}$ of a wave with a source frequency $f_{src}$ can be expressed as:

\begin{equation}
    f_{obs} = \left( \frac{v + v_{obs}}{v - v_{src}} \right) f_{src}
\end{equation}

Where $v$ is the speed of the wave in the medium, $v_{src}$ is the velocity of the source, and $v_{obs}$ is the observer's velocity.
\cite{doppler1842}
\subsection*{Applications}
The Doppler effect has several practical applications in various fields. One of the most well-known applications is in astronomy. By studying the shift in the frequency of light emitted by stars and galaxies, astronomers can determine their relative motion towards or away from Earth. This information provides valuable insights into celestial objects' dynamics and the universe's expansion.

In addition to astronomy, the Doppler effect is utilized in medical imaging techniques such as Doppler ultrasound. By measuring the frequency shift of ultrasound waves reflected from moving objects within the body, doctors can diagnose conditions related to blood flow, such as vascular diseases or abnormalities.

\subsection*{Conclusion}
The Doppler effect is a fundamental concept in wave physics that describes the frequency shift observed when there is relative motion between a wave source and an observer. Its applications span various scientific and technological domains, including astronomy, medical imaging, and more.

\subsection*{Personal Details}
\begin{itemize}
    \item Name: Rutika Gavit 
    \item Roll No: AE22B054 
    \item Github ID: rutikagavit
\end{itemize}

\bibliographystyle{plain}
\bibliography{references}

\end{document}


