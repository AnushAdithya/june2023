\documentclass{article}




\title{Assignment - 4}
\author{Roll No: MM22B067 \\ Name : Vedant Kulkarni \\ Github ID: Vedantkul10}
\date{June 2023}

\begin{document}

\maketitle

\section*{MM22B067}
\textbf{Name of equation: Heisenberg Uncertainty Principle(by Werner Heisenberg in 1927)}
\\ \\
\textbf{Equation:}
\[
\Delta x \cdot \Delta p \geq \frac{\hbar}{2}
\]
Where :
\\
\begin{enumerate}
  \item  \(\Delta x\) :- Uncertainty in the measurement of the position of a particle.
  \item \(\Delta p\) :- Uncertainty in the measurement of the momentum of a particle.
  \item $\hbar = \frac{h}{2\pi}$ .'h' is the Planck's Constant, whose value is $6.62607015 \times 10^{-34} \, \mathrm{J} \cdot \mathrm{s}$.
  \\

\end{enumerate}
\textbf{Reason for taking this equation:}.It is used for study of tiny subatomic particles and to understand how they interact.Hence its application comes under quantum mechanics which according to me is a very interesting topic.
\\ \\
\textbf{Explanation:}
\\
This principle indicates that, although it is possible to measure the momentum or position of a particle accurately, it is not possible to measure these two observables simultaneously to an arbitrary accuracy. This principle is based on the wave-particle duality of matter. \footnote{Reference: Quantum Mechanics: Concepts and Applications by Zettili Nouredine.Chapter 1(Origins of Quantum Physics) , unit 1.5.1( Heisenberg's Uncertainty Principle)}
The more precise our measurement of position is, the less accurate will be our momentum measurement and vice-versa. 
\\ \\
Example : If we find value of momentum very close to the actual value then \\
\\ \(\Delta p \rightarrow 0\) . 
\\ \(\Delta x \geq \frac{h}{4\pi \Delta p}\)
\\ \\
It can be clearly seen that \(\Delta x\) is inversely proportional to \(\Delta p\). Therefore \(\Delta x\) will be tending to \textbf{infinity} for the given value of p.Meaning that the accuracy in finding the position will be very very low.




\end{document}
