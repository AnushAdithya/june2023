\documentclass{article}
\usepackage{graphicx} % Required for inserting images

\title{Poisson's Equation}
\author{Roll no: MM22B021 \\ Name: Babar Girish Sampat \\ GitHub UserId: gbabar64}
\date{June 2023}

\usepackage[paperheight=6in,
   paperwidth=5in,
   top=10mm,
   bottom=20mm,
   left=10mm,
   right=10mm]{geometry}

\begin{document}

\maketitle

\section{Introduction}

\begin{itemize}
    \item Poisson's equation is an elliptic partial differential equation of broad utility in theoretical physics.
    \item For example, the solution to Poisson's equation is the potential field caused by a given electric charge or mass density distribution; with the potential field known, one can then calculate electrostatic or gravitational (force) field.
    \item  It is a generalization of Laplace's equation, which is also frequently seen in physics.
    \item The equation is named after French mathematician and physicist Siméon Denis Poisson.
\end{itemize}

\section{Statement OF Equation}

\begin{itemize}
    \item Poisson's equation is \\\\
\indent\hspace{1cm} $\Delta$$\psi$=$f$, \\
    \item Where $\Delta$ is a Laplace operator, and $f$ and $\psi$ are real or complex-valued functions on a manifold.Usually, the Laplace operator is often denoted as 
$\nabla^2$ , and so Poisson's equation is frequently written as \\\\
\indent\hspace{1cm} $\nabla^2$$\psi$=$f$ \\
    \item In three-dimensional Cartesian coordinates, it takes the form \\\\
\indent\hspace{1cm}$(\frac{\partial^2}{\partial x^2}+\frac{\partial^2}{\partial y^2}+\frac{\partial^2}{\partial z^2})$ $\psi$($x$,$y$,$z$) = $f$($x$,$y$,$z$)
\end{itemize}

\section{Solutions Of Poisson's Equation}
\begin{enumerate}
    \item Poisson's equation for gravity is \\\\
    \indent\hspace{2cm} $\nabla^2V$ =4$\pi$$G$$\rho$
    \item  Poisson's equation in electrostatics is \\\\
    \indent\hspace{2cm} $\nabla^2V$ =-$\frac{\rho}{\epsilon}$\\
    Here,\\
    \indent\hspace{1cm} V = Potential of respective field \\
    \indent\hspace{1cm} G = Universal gravitational constant \\
    \indent\hspace{1cm} $\rho$ = Density of respective field \\
    \indent\hspace{1cm} $\epsilon$ = Permittivity of medium \\
\end{enumerate}


\subsection{Application Of Poisson's Equations}
Poisson's equation is one of the pivotal parts of Electrostatics, where we would solve the equation to find electric potential from a given charge distribution.

\section{Why I Choose Poisson's Equation}

\begin{itemize}
    \item  Poisson's equation is a fundamental partial differential equation that describes the behavior of a scalar potential field.
    \item  Solving Poisson's equation allows us to understand and predict the behavior of physical phenomena like heat conduction, electomagnetism  which will help me in my departmental study in future.
\end{itemize}

\begin{thebibliography}{2}
\bibitem{textbook}
Introduction To Electrodynamics, 3rd Ed. by David J.Griffith
\bibitem{url}
https://en.wikipedia.org/wiki/Poisson%27s_equation
\end{thebibliography}

\end{document}
