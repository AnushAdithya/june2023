\documentclass{article}
\usepackage{url}

\begin{document}

\section{CH22B113}


\subsection{Student Details}

\begin{itemize}
  \item Name: Thiru Kathir
  \item Roll No: CH22B113
  \item GitHub: thiru-kat
\end{itemize}

\subsection{Equation}


The Clausius-Clapeyron equation is a way of describing a discontinuous phase transformation between two phases of matter of a single constituent. It specifies the temperature dependence of pressure, most importantly vapor pressure, at a discontinuous phase transition between two phases of matter of a single constituent. 

\vspace{8mm}
The equation is expressed by:
\begin{equation}
   ln(\frac{P_2}{P_1})=- \frac{\Delta H_{vap}}{R}(\frac{1}{T_2}-\frac{1}{T_1})
\end{equation}

where,
\begin{itemize}
    \item $\Delta H_{vap}$= Enthalpy of vaporisation
    \item $T_2,T_1$ = Temperatures in Kelvin 
    \item $P_2,P_1$ = Pressures at $T_1$ and $T_2$ respectively
    \item $R$ = Universal Gas Constant = 8.314 J/ mol-K
\end{itemize}




The Clausius-Clapeyron equation predicts the rate at which vapor pressure increases per unit increase in temperature for a substance’s vapor pressure (P) and temperature (T). According to this equation the vapour pressure of a system increases much faster than the system's temperature. It allows us to estimate the vapor pressure at another temperature if the vapor pressure is known at some temperature and if the enthalpy of vaporization is known.

The Clausius-Clapeyron equation applies to vaporization of liquids where vapor follows ideal gas law using the ideal gas constant and liquid volume is neglected as being much smaller than vapor volume V. It is often used to calculate vapor pressure of a liquid.


\footnote{\url{https://www.vedantu.com/physics/clausius-clapeyron-equation}}
\footnote{\url{https://en.wikipedia.org/wiki/Clausius-Clapeyron_relation}}


\end{document}
