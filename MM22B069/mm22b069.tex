\documentclass{article}
\usepackage{graphicx} 



\title{Assignment 4}
\author{Roll No : MM22B069 \\ Name : Vishwajeet Nitin Pawar mm22b069 \\ Github ID: Vishu200412}
\date{June 2023}

\begin{document}

\maketitle
\begin{center}
    
\end{center}
\section*{MM22B069}
\subsection*{Newton's Second Law of Motion}
Newton's second law gives the cause-and-effect relationship between force and changes in motion. The acceleration of a system is directly proportional to and in the same direction as the net external force acting on the system and is inversely proportion to its mass \footnote{Rob Pryce. “4.3 Newton’s Second Law”. In: Introduction to Biomechanics
(2022).}.

\noindent 
In equation form, Newton’s second law is written as:
\[\sum \vec{F} = \vec{F}_{\text{net}} = ma\]
\subsection*{Derivation: Newton’s Second Law and Momentum}
Newton stated his second law in terms of momentum: “The instantaneous rate at which a body’s momentum changes is equal to the net force acting on the body.” (“Instantaneous rate” implies that the derivative is involved.) This can be given by the vector equation:
\[\vec{F}_{\text{net}} = \frac{d\vec{p}}{dt}\]
The momentum \(\vec{p}\) is defined as the product of the mass of the object \(m\) and its velocity \(\vec{v}\).
\[\vec{p} = m\vec{v}\]
\[\vec{F}_{\text{net}} = \frac{d\vec{p}}{dt} = \frac{d(m\vec{v})}{dt}\]
When m is constant, we have:
\[F_{\text{net}} = m\frac{d\vec{v}}{dt} = m\vec{a}\]
\section*{Reason for choosing the equation}
I chose Newton's second law because it is widely applicable in physics and engineering and forms the basis for analyzing and predicting the motion of objects.


\end{document}