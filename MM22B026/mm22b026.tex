\documentclass{article}
\usepackage{caption}
\usepackage{float}
\usepackage{graphicx}
\usepackage{amsmath} % Required for inserting images
\graphicspath{ {./images/} }
\title{Assignment 4}
\author{\\ Name : Ish Haresh Gavkhadkar \\ Github ID : ishgavkhadkar}
\date{}
\begin{document}
\maketitle
\section*{MM22B026}
\subsection*{PROJECTILE MOTION}
\textbf{Introdution:-}
\\
\\
\textbf{Projectile motion},refers to the motion of an object launched
into the air that follows a curved path under the influence of gravity,
with no other forces acting on it.
\\
Q1) Department: an equation, a plot that describes the equation, a brief
description of the equation and the symbols contained in it, a set of
references that give more informatiom about the equation and finally, a
paragraph on how you went about choosing the equation.
\\
\\
\textbf{A. Vertical motion of a projectile under the influence of gravity.}
\\
\\
\textbf{Equation:}\(\ \mathbf{y = u}\mathbf{\sin}\mathbf{\text{θt}}\mathbf{-}\frac{\mathbf{1}}{\mathbf{2}}\mathbf{g}\mathbf{t}^{\mathbf{2}}\mathbf{+ u}\mathbf{\sin}{\mathbf{\theta}\frac{\mathbf{t}_{\mathbf{\text{flight}}}}{\mathbf{2}}}\)

\begin{itemize}
\item
  \textbf{y:} Vertical position of the projectile at a given time (m)
\item
  \textbf{u}: Initial velocity of the projectile (m/s)
\item
  \textbf{θ:} Launch angle of the projectile (deg)
\item
  \textbf{t:} Time (s)
\item
  \textbf{g:} Acceleration due to gravity (9.8 m/s\^{}2)
\item
  \(\mathbf{t}_{\mathbf{\text{flight}}}\)\textbf{:} Time of flight of
  the projectile (s)
\end{itemize}

\begin{itemize}
\item
  \textbf{y} represents the vertical position of the projectile at a
  given time. It tells us how high or low the projectile is above the
  ground (measured in meters).
\item
  \textbf{u} is the initial velocity of the projectile. It represents
  the magnitude of the velocity vector at the instant of launch. The
  velocity has both horizontal and vertical components.
\item
  \textbf{sin θ} is the sine of the launch angle θ. The launch angle
  determines the direction of the projectile's initial velocity vector.
  It is measured in degrees.
\item
  \textbf{t} is the time elapsed since the projectile was launched. It
  represents the independent variable of the equation and is measured in
  seconds.
\item
  \textbf{g} is the acceleration due to gravity. On Earth, it is
  approximately 9.8 (m/s\^{}2). It affects the vertical motion of the
  projectile, causing it to accelerate downward.
\end{itemize}

\begin{itemize}
\item
  \(\mathbf{t}_{\mathbf{\text{flight}}}\)represents the total time of
  flight of the projectile. It is the time it takes for the projectile
  to reach the ground after being launched. It is calculated as 2 × u ×
  sin \textbf{θ} / g.
\end{itemize}
\begin{enumerate}
\def\labelenumi{\arabic{enumi}.}
\item
  \(\mathbf{u}\mathbf{\sin}\mathbf{\text{θt}}\)\textbf{:} This term
  represents the initial upward velocity component of the projectile. As
  the projectile is launched, this component contributes to the vertical
  displacement. It gradually decreases over time due to the effect of
  gravity.
\item
  \(\frac{\mathbf{1}}{\mathbf{2}}\mathbf{g}\mathbf{t}^{\mathbf{2}}\mathbf{:}\)This
  term represents the vertical displacement due to the downward
  acceleration caused by gravity. As time increases, the effect of
  gravity becomes more pronounced, causing the projectile to fall
  faster.
\item
  \(\mathbf{u}\mathbf{\sin}\mathbf{\theta}\frac{\mathbf{t}_{\mathbf{\text{flight}}}}{\mathbf{2}}\)\textbf{:}
  This term adjusts the equation to ensure the final position matches
  the initial position. It compensates for the fact that the projectile
  starts and ends at the same vertical position.
\end{enumerate}
\\
\\
\textbf{B. Horizonal motion of a projectile under the influence of gravity}
\\
\\
\textbf{Equation:-}
\(\mathbf{x = u}\mathbf{\cos}\mathbf{\text{θt}}\mathbf{+ u}\mathbf{\sin}\mathbf{\theta}\frac{\mathbf{t}_{\mathbf{\text{flight}}}}{\mathbf{2}}\)

\begin{itemize}
\item
  \textbf{x:} Horizontal position of the projectile at a given time (m)
\item
  \textbf{u}: Initial velocity of the projectile (m/s)
\item
  \textbf{θ}: Launch angle of the projectile (deg)
\item
  \textbf{t:} Time (s)
\item
  \textbf{g:} Acceleration due to gravity (9.8 m/s\^{}2)
\item
  \(\mathbf{t}_{\mathbf{\text{flight}}}\)\textbf{:} Time of flight of
  the projectile (s), font=("Arial", 25)
\end{itemize}

\begin{itemize}
\item
  \textbf{x} represents the horizontal position of the projectile at a
  given time. It tells us how far the projectile has traveled
  horizontally from the starting point (measured in meters).
\item
  \textbf{u} is the initial velocity of the projectile. It represents
  the magnitude of the velocity vector at the instant of launch. The
  velocity has both horizontal and vertical components.
\item
  \textbf{cos θ:} is the cosine of the launch angle θ .The launch angle
  determines the direction of the projectile's initial velocity vector.
  It is measured in degrees.
\item
  \textbf{t} is the time elapsed since the projectile was launched. It
  represents the independent variable of the equation and is measured in
  seconds.
\item
  \(\mathbf{t}_{\mathbf{\text{flight}}}\)\textbf{:}represents the total
  time of flight of the projectile. It is the time it takes for the
  projectile to reach the ground after being launched. It is calculated
  as 2 × u × sin(θ) / g, where g is the acceleration due to gravity.
\end{itemize}

\begin{enumerate}
\def\labelenumi{\arabic{enumi}.}
\item
  \(\mathbf{u}\mathbf{\cos}\mathbf{\text{θt}}\) This term represents the
  horizontal displacement of the projectile due to its initial velocity
  component in the x-direction. It describes the projectile's horizontal
  motion without considering the effect of gravity.
\item
  \(\mathbf{u}\mathbf{\sin}\mathbf{\theta}\frac{\mathbf{t}_{\mathbf{\text{flight}}}}{\mathbf{2}}\)\textbf{:}
  This term adjusts the equation to ensure the final position matches
  the initial position. It compensates for the fact that the projectile
  starts and ends at the same horizontal position.
\end{enumerate}
\\
\\
\textbf{C. Time of Flight :-}
\\
\\
\textbf{Equation:-}
\(\mathbf{t}_{\mathbf{\text{flight}}}\mathbf{=}\frac{\mathbf{2}\mathbf{u}\mathbf{\sin}\mathbf{\theta}}{\mathbf{g}}\)

\begin{itemize}
\item
  \(\mathbf{t}_{\mathbf{\text{flight}}}\)\textbf{:} Time of flight of
  the projectile (s)
\item
  \textbf{u}: Initial velocity of the projectile (m/s)
\item
  \textbf{θ:} Launch angle of the projectile (deg)
\item
  \textbf{g:} Acceleration due to gravity (9.8 m/s\^{}2)
\item
  \(\mathbf{t}_{\mathbf{\text{flight}}}\mathbf{\ }\)represents the time
  of flight of the projectile, which is the total time it spends in the
  air from the moment of launch until it lands. It is measured in
  seconds.
\item
  \textbf{u} is the initial velocity of the projectile. It represents
  the magnitude of the velocity vector at the instant of launch. The
  velocity has both horizontal and vertical components.
\item
  \textbf{sinθ} is the sine of the launch angle . The launch angle
  determines the direction of the projectile's initial velocity vector.
  It is measured in degrees.
\item
  \textbf{g} is the acceleration due to gravity, which is approximately
  9.8 m/s\^{}2. It represents the downward acceleration experienced by
  the projectile.
\end{itemize}

\begin{enumerate}
\def\labelenumi{\arabic{enumi}.}
\item
  The equation itself is derived from the kinematic equations of motion
  for projectile motion
\item
  By plugging in the values of the initial velocity and launch angle,
  the equation can be used to calculate the time of flight for a given
  projectile.
\end{enumerate}
\\
\\
\textbf{D. Horizontal range of a projectile launched at a given initial
velocity and angle}
\\
\\
\textbf{Equation:}
\(\text{}\mathbf{R =}\frac{\mathbf{u}_{\mathbf{0}}^{\mathbf{2}}\mathbf{\sin}{\mathbf{2}\mathbf{\theta}}}{\mathbf{g}}\)

\begin{itemize}
\item
  \textbf{R}: Horizontal range of a projectile launched (m)
\item
  \textbf{u}: Initial velocity of the projectile (m/s)
\item
  \textbf{θ:} Launch angle of the projectile (deg)
\item
  \textbf{g:} Acceleration due to gravity (9.8 m/s\^{}2)
\end{itemize}

\begin{itemize}
\item
  \textbf{R} represents the horizontal range of the projectile, which is
  the horizontal distance traveled by the projectile before it lands. It
  is measured in meters.
\item
  \textbf{u} is the initial velocity of the projectile. It represents
  the magnitude of the velocity vector at the instant of launch. The
  velocity has both horizontal and vertical components.
\item
  \textbf{θ} is the launch angle of the projectile. It determines the
  direction of the projectile's initial velocity vector. It is measured
  in degrees.
\item
  \textbf{sin2θ} is the sine of twice the launch angle. The factor of 2
  accounts for the symmetrical nature of the projectile's trajectory.
\item
  \textbf{g} is the acceleration due to gravity, which is approximately
  9.8 m/s\^{}2. It represents the downward acceleration experienced by
  the projectile
\end{itemize}

\textbf{Q2) Paragraph on how you went about choosing the equation.}
\\
\\
Firstly, I choosed this equation, because we are dealing with projectile
motion since class 11 and its interesting and easy to understand.
Projectile motion refers to the path followed by an object in motion
under the influence of gravity, where the only force acting on it is the
force of gravity. To accurately describe this motion, I chose the
equation that consists elements: initial velocity, launch angle, time of
flight, horizontal and vertical displacements, and acceleration due to
gravity. Moreover, working with projectile motion equations provided me
with a practical toolkit for solving real-world problems. Whether it was
analyzing the trajectory of a launched rocket, understanding the motion
of a baseball, or predicting the range of a cannonball, I felt empowered
to apply my knowledge and make meaningful calculations. It make us to
think pratically. In conclusion, I have chosen parametric equations for
projectile motion because they provide a comprehensive framework for
understanding and analyzing the behavior of objects moving under the
influence of gravity.
\\
\\
\textbf{Conclusion:-}
\\
\\
In conclusion, projectile motion is a fascinating phenomenon that occurs
when an object is launched into the air and moves under the influence of
gravity alone. It involves the object following a parabolic trajectory,
with its motion being divided into separate horizontal and vertical
components. By understanding and applying the principles of projectile
motion, we can analyze and predict the behavior of objects those launched at different angles and velocities

\bibliography{refs}
\bibliographystyle{alpha}
\footnote{Wikipedia,"Article:Projectile Motion",(Accessed: June , 2023)}
\end{document}
