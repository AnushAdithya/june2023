\documentclass{article}
\usepackage{graphicx} % Required for inserting images

\title{ID2090-Assignment-4}
\author{M Mohnish}
\date{June 2023}

\begin{document}

\maketitle
\newpage

\section{Introduction}



The Schrödinger wave equation is a fundamental equation in quantum mechanics that describes the behavior of wavefunctions. It was first formulated by Erwin Schrödinger in 1925 and has since played a central role in understanding the quantum world.

\section{Equation}

The time-dependent Schrödinger wave equation is given by:

\begin{equation}
i\hbar \frac{\partial \Psi}{\partial t} = -\frac{\hbar^2}{2m} \nabla^2 \Psi + V \Psi
\end{equation}

where:
\begin{itemize}
  \item $i$ is the imaginary unit,
  \item $\hbar$ (h-bar) is the reduced Planck's constant ($\hbar = \frac{h}{2\pi}$),
  \item $\frac{\partial \Psi}{\partial t}$ represents the time derivative of the wavefunction $\Psi$,
  \item $\nabla^2 \Psi$ represents the Laplacian operator applied to the wavefunction $\Psi$,
  \item $m$ is the mass of the particle, and
  \item $V$ represents the potential energy of the system.
\end{itemize}

\section{Solutions}

Solving the Schrödinger wave equation involves finding the eigenvalues and eigenfunctions of the equation. The time-independent Schrödinger equation, obtained by separating the time variable, is often used to find stationary states and energy eigenvalues.

The solutions to the Schrödinger wave equation, represented by the wavefunction $\Psi$, provide a complete description of the quantum state of a system. The wavefunction contains information about the probabilities of different measurement outcomes and the spatial distribution of the system.

\section{Applications}

The Schrödinger wave equation has been successfully used to describe a wide range of physical phenomena. Some of its applications include:

\begin{itemize}
  \item Describing the behavior of electrons in atoms and the formation of atomic orbitals.
  \item Explaining the bonding and molecular orbitals in chemistry.
  \item Understanding the dynamics of quantum particles in various potentials, such as the harmonic oscillator, the particle in a box, and the hydrogen atom.
\end{itemize}

\section{Conclusion}

The Schrödinger wave equation is a fundamental equation in quantum mechanics that provides a mathematical framework for understanding the wave-like nature of particles and their behavior in quantum systems. It has profound implications and has been essential in explaining various quantum phenomena.

\section{Bibliography}
\cite{beiser1963}

\bibliographystyle{plain}
\bibliography{references}





\end{document}

\end{document}
