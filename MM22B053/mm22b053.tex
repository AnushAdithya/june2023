\documentclass{article}
\usepackage{graphicx} % Required for inserting images

\title{ID2090 Assignment 4}
\author{Rishi Kannan }
\date{MM22B053}

\begin{document}

\maketitle

\section{}
\subsection{General}

Name : Rishi Kannan\\
Github ID : quacknutzz


\subsection{Wilson's Theorem equation}

\large Wilson's theorem is the theorem that revolutionised how we work with Prime Numbers. 

Wilson's theorem, in the realm of algebra and number theory, asserts that for a natural number n greater than 1, it is classified as a prime number when the product of all positive integers less than n is precisely one less than a multiple of n.\\
If p is a prime number, then

\footnote{Reference for equation : YouTube - Michael Penn}
\Large$(p-1)! \equiv -1\pmod{p}$\\ 
\large In other words, p is prime if and only if (p-1)! + 1 is divisible by p. 

\subsection{History of the equation}

\large Ibn al-Haytham originally stated this theorem in 1000 AD, while John Wilson reiterated it in the 18th century. Edward Waring made the theorem public in 1770, but neither he nor his student Wilson were able to provide a proof for it. Subsequently, Lagrange presented the first proof of the theorem in 1771.\\
It has been proved using both composite and prime modulus methods. The prime modulus method largely consists of \textit{Fermat's little theorem} and \textit{Sylow theorems}.

\subsection{Applications of Wilson's Theorem}

\large Wilson's Theorem has been been largely used in 
\begin{itemize}
  \item Primality tests
  \item Quadratic residues
  \item p-adic gamma function
  \item Wilson's Theorem is used in cryptography for coding-decoding.
\end{itemize}




\end{document}
