\documentclass{article}
\usepackage{amsmath}

\begin{document}

\title{Ampere's Circuital Law}
\date{\today}

\maketitle
\section{MM22B008}
NAME: NAVEEN M
\newline
GITHUB ID: Naveen1065
\subsection{Introduction}
Ampere's circuital law is a fundamental principle in electromagnetism that relates the magnetic field produced by a current to the current itself. It was first formulated by the French physicist André-Marie Ampère in the early 19th century.

\subsection{Equation}
The mathematical expression for Ampere's circuital law is given as follows:

\begin{equation}
\oint \mathbf{B} \cdot d\mathbf{l} = \mu_0 \iint \mathbf{J} \cdot d\mathbf{A}
\end{equation}
where:
\begin{align*}
\oint & \text{ denotes the closed line integral around a loop}\\
\mathbf{B} & \text{ is the magnetic field vector}\\
d\mathbf{l} & \text{ is an infinitesimal vector element along the loop}\\
\mu_0 & \text{ is the permeability of free space}\\
\mathbf{J} & \text{ is the current density vector}\\
d\mathbf{A} & \text{ is an infinitesimal vector element of a surface bounded by the loop}
\end{align*}

\footnotetext{www.wikipedia.org}

\subsection{Conclusion}
Ampere's circuital law provides a powerful tool for calculating the magnetic field around current-carrying conductors. It is widely used in various applications, including the design of electromagnets, transformers, and motors.

\end{document}
