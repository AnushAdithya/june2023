\documentclass[12pt,a4paper]{article}
\usepackage{amsmath}
\usepackage{amsfonts}
\usepackage{amssymb}
\usepackage{graphicx}


\title{Assignment 4 }
\author{Rushikesh Shankarrao Kapale CH22B023}
\date{June 2023}
\begin{document}
\maketitle
\begin{center}
    USER ID: rush9970
\end{center}
\title{The Gibbs-Helmholtz Equation}

\date{\today}



\maketitle

\section{The Gibbs-Helmholtz Equatio}
The Gibbs-Helmholtz equation is a thermodynamic relationship that relates the change in Gibbs free energy (\(\Delta G\)) with temperature (\(T\)). It provides  the temperature dependence of the spontaneity of chemical reactions. 

\section{Formula}
The Gibbs-Helmholtz equation is given by~\cite{Atkins}:

\begin{equation}
\left(\frac{\partial (\Delta G / T)}{\partial T}\right)_P = -\frac{\Delta H}{T^2}\label{eqn:atkins}
\end{equation}

where \(\Delta G\) is the change in Gibbs free energy, \(\Delta H\) is the change in enthalpy, \(T\) is the temperature, and \((\partial (\Delta G / T) / \partial T)_P\) denotes the derivative of \(\Delta G / T\) with respect to temperature at constant pressure.The Gibbs-Helmholtz equation allows us to determine the spontaneity of a chemical reaction based on the temperature dependence of the Gibbs free energy change. 

In this document we refer book Atkins Physical Chemistry


\bibliography{reff}
\bibliographystyle{plain}

\end{document}
