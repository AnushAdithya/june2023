\documentclass{article}
\usepackage{amsmath}
\usepackage{graphicx}

\title{Conservation of Momentum}
\author{Balbhadra Tubid   
Roll no-Ch22b040}

\begin{document}

\maketitle

\section{Introduction}
The principle of conservation of momentum states that the total momentum of an isolated system remains constant if no external forces act on it. This principle is widely applicable in various fields, including physics and engineering.

\section{Conservation of Momentum in Collisions}
In collisions, momentum is conserved. There are three types of collisions: perfectly elastic, perfectly inelastic and inelastic.


\subsection{Perfectly Elastic Collisions}
In an elastic collision, both momentum and kinetic energy are conserved. This means that the total momentum before the collision is equal to the total momentum after the collision, and the total kinetic energy before the collision is equal to the total kinetic energy after the collision.

Mathematically, for a two-object elastic collision, we have the following equations~\cite{arie}

\begin{align}
m_1 v_{1i} + m_2 v_{2i} &= m_1 v_{1f} + m_2 v_{2f} \\
\frac{1}{2} m_1 v_{1i}^2 + \frac{1}{2} m_2 v_{2i}^2 &= \frac{1}{2} m_1 v_{1f}^2 + \frac{1}{2} m_2 v_{2f}^2
\end{align}

where $m_1$ and $m_2$ are the masses of the two objects, $v_{1i}$ and $v_{2i}$ are their initial velocities, and $v_{1f}$ and $v_{2f}$ are their final velocities.

\subsection{Perfectly Inelastic Collisions}
In an inelastic collision, only momentum is conserved, while kinetic energy is not necessarily conserved. This means that the total momentum before the collision is equal to the total momentum after the collision, but the total kinetic energy may change.

Mathematically, for a two-object inelastic collision, we have the following equation:~\cite{arie}

\begin{equation}
m_1 v_{1i} + m_2 v_{2i} = (m_1 + m_2) v_f
\end{equation}

where $m_1$ and $m_2$ are the masses of the two objects, $v_{1i}$ and $v_{2i}$ are their initial velocities, and $v_f$ is their final velocity.

\subsection{Non-elastic Collisions}
In a non-elastic collision, momentum is conserved, but the kinetic energy is not conserved. Instead, a coefficient of restitution, denoted as $e$, is introduced to represent the ratio of final relative velocity to the initial relative velocity.

Mathematically, for a two-object non-elastic collision, we have the following equation:~\cite{arie}

\begin{equation}
v_{1f} - v_{2f} = e(v_{1i} - v_{2i})
\end{equation}

where $v_{1i}$ and $v_{2i}$ are the initial velocities, $v_{1f}$ and $v_{2f}$ are the final velocities, and $e$ is the coefficient of restitution.






\section{Conclusion}
The conservation of momentum is a fundamental principle in physics that helps explain the behavior of objects in collisions. In elastic collisions, both momentum and kinetic energy are conserved, while in inelastic collisions, only momentum is conserved. These concepts are essential in understanding and analyzing the outcomes of various physical interactions.



\bibliography{refs}
\bibliographystyle{plain}

\end{document}


